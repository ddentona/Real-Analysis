\documentclass[12pt]{report}
\usepackage{amsmath}
\usepackage{amssymb}
\usepackage{amsthm}
\usepackage{multicol}

\begin{document}
\tableofcontents
\pagebreak
\chapter{Keeping it Real}
\section{Exercise 1.1}
Explain the error in the following ``proof'' that 2 = 1.\\
Let $x = y$. Then
\begin{align*}
    x^2 &= xy\\
    x^2 + y^2 &= xy - y^2\\
    (x + y)(x - y) &= y(x - y)\\
    x + y &= y\\
    2y &= y\\
    2 &= 1
\end{align*}

\subsection*{Solution}
Between \((x + y)(x - y) = y(x - y)\) and \(x + y = y\), we divide by \(x - y\), which from the given, is equal to zero. Hence, the step involves dividing by zero, itself undefined and not allowed to be performed.

\pagebreak
\section{Exercise 1.2}
Which of the following statements are true? Give a short explanation for each of your answers?\\
(a) For every $n \in \mathbb{N}$ there is an $m \in \mathbb{N}$ such that $m > n$.\\
(b) For every $m \in \mathbb{N}$ there is an $n \in \mathbb{N}$ such that $m > n$.\\
(c) There is an $m \in \mathbb{N}$ such that for every $n \in \mathbb{N}$, $m \ge n$.\\
(d) There is an $n \in \mathbb{N}$ such that for every $m \in \mathbb{N}$, $m \ge n$.\\
(e) There is an $n \in \mathbb{R}$ such that for every $m \in \mathbb{R}$, $m \ge n$.\\
(f) For every pair $x < y$ of integers, there is an integer $z$ such that $x < z < y$.\\
(g) For every pair $x < y$ of real numbers, there is a real number $z$ such that $x < z < y$.

\subsection{Solution (a) True}
\begin{proof}
    We begin with our given, which is that $n \in \mathbb{N}$. 
    Natural numbers have a quality that they are ordered such that successive numbers are a distance of 1 apart.
    Natural numbers are also unlimited in the direction of positive infinity.
    This means that for every $n \in \mathbb{N}$, there is a greater natural number.
    In other words, for every $n \in \mathbb{N}$, there exists an $m \in \mathbb{N}$ such that $m > n$.
\end{proof}

\subsection{Solution (b) False}
To prove this false, we need but find a counterexample. 
For the counterexample, we look at the number 1. $1 \in \mathbb{N}$. However, there is no $n \in \mathbb{N}$ such that $1 > n$. As such, it is false.

\subsection{Solution (c) False}
There is not maximum number in the natural numbers. For this to be true, it would have to contradict part (a), which has been proven. As such it is false.

\subsection{Solution (d) True}
The contrapositive of this is there is no $n \in \mathbb{N}$ such that for every $m \in \mathbb{N}$, $m < n$. Since there are infinitely many $\mathbb{R}$ with no upward bound, there will be no number $n \in \mathbb{N}$ such that $m < n$ for every $m \in \mathbb{N}$. This makes it true.

\subsection{Solution (e) True}
The contrapositive of this is there is no $n \in \mathbb{R}$ such that for every $m \in \mathbb{R}$, $m < n$. Since there are infinitely many $\mathbb{R}$ with no upward bound, there will be no number $n \in \mathbb{R}$ such that $m < n$ for every $m \in \mathbb{R}$. This makes it true.

\subsection{Solution (f) False}
If $x$ and $y$ are adjacent, like 0 and 1, then the number $z$ such that $x < z < y$ would have to be of distance less than 1 from the two adjacent numbers. This contradicts the principle of integers, which says that all integers are of distance 1 from their adjacent numbers. The statement is false.

\subsection{Solution (g) True}
There are considered to be infinitely many real numbers between two other real numbers.
This means that for any $x, y \in \mathbb{R}$, there does exist $z \in \mathbb{R}$ such that $x < z < y$. 

\pagebreak
\section{Exercise 1.3}
If $A$ and $B$ are two boxes (possibly with things inside), describe the following in terms of boxes:
\begin{align*}
    &(a) A \setminus B 
    &(b) \mathcal{P}(A)
    &&(c) \left| A \right|
\end{align*} 

\subsection{Solution (a)}
$A \setminus B$ means the set of objects in A not in B. 
$A \setminus B = \{x: x \in A \text{and} x \notin B\}$. 
In terms of boxes, this would mean the box with everything that is in A but not in B.

\subsection{Solution (b)}
$\mathcal{P}(A)$ or the \textit{power set} of A means the set of all subsets of A.
\(\mathcal{P}(A) = \{X: X \subseteq B\}\).
In terms of boxes, this would be the box containing every box containing a combination of only different objects contained in $A$. 

\subsection{Solution (c)}
$\left| A \right|$ or the cardinatity of A means the count of all elements in $A$.
In terms of boxes, this would be the number of objects contained in the box $A$.

\pagebreak
\section{Exercise 1.4}
If $A_1, A_2, A_3, \dots, A_n$ are all boxes (possibly with things inside), describe the following in terms of boxes:
\begin{align*}
    &(a) \overunderset{n}{i = 1}{\bigcup} A_i = A_1 \cup A_2 \cup \dots \cup A_n
    &(b) \overunderset{n}{i = 1}{\bigcap} A_i = A_1 \cap A_2 \cap \dots \cap A_n
\end{align*}

\subsection{Solution (a)}
A box containing all the objects in each of the boxes $A_1, A_2, A_3, \dots, A_n$.

\subsection{Solution (b)}
A box containing each object that is in every one of the boxes $A_1, A_2,$ $A_3, \dots, A_n$.

\pagebreak
\section{Exercise 1.5}
Prove that each of the following holds for any sets $A$ and $B$.\\
(a) $A \cup B = A$ if and only if $B \subseteq A$.\\
(b) $A \cap B = A$ if and only if $A \subseteq B$.\\
(c) $A \setminus B = A$ if and only if $A \cap B = \emptyset$.\\
(d) $A \setminus B = \emptyset$ if and only if $A \subseteq B$.

\subsection{Solution (a)}
We have to prove $A \cup B = A$ if and only if $B \subseteq A$.
To prove this, we must prove that $A \cup B = A \implies B \subseteq A$ and $B \subseteq A \implies A \cup B = A$.
We begin with the first. 
\begin{proof}
    We know that $A \cup B = A$.
    This means that for any $b \in B$ or $b \in A$, $b \in A$. 
    Taking the first half of this, we get $b \in B \implies b \in A$.
    This can be rewritten as $B \subseteq A$. 
    QED
\end{proof}

Now we prove the second.
\begin{proof}
    We know that $B \subseteq A$.
    This can be written that $b \in A$ for every $b \in B$, or $\frac{b \in B}{b \in A}$.\\
    From identity, we also know that $b \in A$ for every $b \in A$, or $\frac{b \in A}{b \in A}$.\\
    Putting these together, we get $b \in A$ for every $b \in A$ or $b \in B$, rewriten as $\frac{(b \in A) \vee (b \in B)}{b \in A}$.\\
    Rewriting this in sets, we get $\frac{\{b : b \in A \text{ or } b \in B\}}{\{b: b \in A\}}$.\\
    Putting this in implies and sets notation, we finally end up with \(A \cup B = A\). \\
    QED
\end{proof}
Putting these together, we get the if and only if statement we were looking for. 

\subsection{Solution (b)}
Part 1: $A \subseteq B \implies A \cap B = A$
\begin{proof}
    (1) It is given that $A \subseteq B$. 
    From this, we know from definitional that for any $a \in A$, $a \in B$.
    \[a \in A \implies a \in B\].\\
    (2) In what I call ``The 30s conjecture'' (iff a person is in their thirties then they are in their thirties and are in their thirties), we can say that if and only if $a \in A$ and $a \in A$, then $a \in A$.
    \[A = \left\{a: a \in A, a \in A\right\}\]
    (3) From step (1), we can put in the implication into part (2) that if and only if $a \in A$ and $a \in B$, then $a \in A$.
    \[A = \left\{a: a \in A, a \in B\right\}\]
    (4) We can rewrite that as $A \cap B = A$.
    QED
\end{proof}

\newcommand{\andtxt}{\text{ and }}
Part 2: $A \cap B = A \implies A \subseteq B$
\begin{proof}
    Suppose that $A \cap B = A$.
    We can establish definitions for this.
    \begin{gather}
        A \cap B = \left\{x: x \in A \andtxt x \in B\right\}
    \end{gather}

    From the transistive property, we can continue.
    \begin{equation}
        A = \left\{x: x \in A \andtxt x \in B\right\}
    \end{equation}

    We can then establish the reflexive definition for the set A and apply that to the above equation with the transistive property.
    \begin{gather}
        A = \left\{ x: x \in A \right\}\\
        \left\{ x: x \in A \right\} = \left\{x: x \in A \andtxt x \in B\right\}
    \end{gather}

    We can use this to define a generic element $x$.
    \begin{gather}
        x \in A \Leftrightarrow x \in A \andtxt x \in B\\
        x \in A \implies x \in A \andtxt x \in B
    \end{gather}

    This can be separated into two implications.
    \begin{gather}
        x \in A \implies x \in A\\
        x \in A \implies x \in B
    \end{gather}

    We can turn the second point into $A \subseteq B$.\\
    Q.E.D.
\end{proof}

\subsection{Solution (c)}
Part 1: $A \cap B = \emptyset \implies A \setminus B = A$
\begin{proof}
    If $A \cap B = \emptyset$, that means that $a \notin B$ for any $a \in A$ and $a \notin A$ for any $a \in B$.
    It also means that $A$ and $B$ are disjoint.
    \begin{gather}
        a \in B \implies a \notin A\\
        a \in A \implies a \notin B
    \end{gather}
    
    We can also define the set $A \setminus B$.
    \begin{equation}
        A \setminus B = \left\{ x: x \in A, x \notin B \right\}
    \end{equation}
    
    We can set the falsehood to its equivalent.
    \begin{equation}
        A \setminus B = \left\{ x: x \in A, \overline{x \in B} \right\}
    \end{equation}

    The equivalent takes in (1.2).
    \begin{equation}
        A \setminus B = \left\{ x: x \in A, \overline{x \notin A} \right\}
    \end{equation}

    We can turn the double negative into a positive.
    \begin{equation}
        A \setminus B = \left\{ x: x \in A, x \in A \right\}
    \end{equation}

    We can then combine the two and compare then to the identity.
    \begin{gather}
        A \setminus B = \left\{ x: x \in A \right\}\\
        A = \left\{x : x \in A\right\}\\
        A \setminus B = A
    \end{gather}
    QED
\end{proof}

Part 2: $A \setminus B = A \implies A \cap B = \emptyset$
\begin{proof}
    Given: $A \setminus B = A$.
    \begin{gather}
        x \in A \land x \notin B \leftrightarrow x \in A
    \end{gather}
    
    We define the set $A \cap B$.
    \begin{gather}
        A \cap B = \left\{x: x \in A, x \in B\right\}
    \end{gather}

    We can substtute in the equivalence (1.10) into (1.11).
    \begin{gather}
        A \cap B = \left\{x: x \in A \land x \notin B, x \in B\right\}\\
        A \cap B = \left\{x: x \in A, x \notin B, x \in B\right\}
    \end{gather}

    This then gives us a contradiction. An element cannot both be in and not in set $B$. 
    As such, we can equate the equivalence with an empty set.
    \begin{gather}
        A \cap B = \emptyset
    \end{gather}
    QED
\end{proof}

\subsection{Solution (d)}
Part 1: $A \subseteq B \implies A \setminus B = \emptyset$.
\begin{proof}
    We can define the relationship $A \subseteq B$ and the set $A \setminus B$.
    \begin{gather}
        x \in A \implies x \in B\\
        A \setminus B = \left\{x:x \in A, x \notin B\right\}
    \end{gather}

    We can substitute (1.15) into (1.16).
    \begin{equation}
        A \setminus B = \left\{x: x \in B, x \notin B\right\}
    \end{equation}

    This generates a contradiction, which we can use to turn the set into a null set.
    \begin{equation}
        A \setminus B = \emptyset
    \end{equation}
    QED
\end{proof}

Part 2: $A \setminus B = \emptyset \implies A \subseteq B$.
\begin{proof}[Proof by Contradiction]
    For the sake of contradiction, we assume that $A \nsubseteq B$.
    We also have as given, that $A \setminus B = \emptyset$.
    \begin{gather}
        A \setminus B = \emptyset\\
        A \setminus B = \left\{x: x \in A, x \notin B\right\}
    \end{gather}
    We can infer from this the existence of an element $x$.
    \begin{gather}
        x \in A\\
        x \notin B
    \end{gather}
    Based on the definition, it would follow that $x$ would be an element of $A \setminus B$.
    We can apply a transistive property to this.
    \begin{gather}
        x \in A \setminus B\\
        x \in \emptyset
    \end{gather}
    Since $\emptyset$ has no elements, it contradicts its definition to say that $x$ is an element of $\emptyset$.
    As such, our assumtion cannot be true. 
    This valudates that $A \setminus B = \emptyset \implies A \subseteq B$.\\
    Q.E.D.
\end{proof}

\pagebreak
\section{Exercise 1.6}
Suppose f: $X \rightarrow Y$ and $A \subseteq X$ and $B \subseteq Y$.\\
(a) Prove that $f(f^{-1}(B)) \subseteq B$.\\
(b) Give an example where $f(f^{-1}(B)) \neq B$.\\
(c) Prove that $A \subseteq f^{-1}(f(A))$.\\
(d) Give an example where $A \neq f^{-1}(f(A))$.

\subsection{Solution (a)}
\begin{proof}
    We can set a reflexive truth involving $B$.
    \begin{equation}
        B = B
    \end{equation}

    Since $f: X \rightarrow Y$ and $B \subseteq Y$, we can say that $B$ is part of the codomain of $f$.
    We can as such apply the inverse function of $f$ to both sides of the reflexive property.
    \begin{equation}
        f^{-1}(B) = f^{-1}(B)
    \end{equation}

    Lastly, we can apply the function $f$ to both sides, which may cancel out on the right side. 
    Since there may be some values in $b \in B$ for which $f^{-1}(B)$ is undefined or does not exist, we can no longer set it as an equality. However, se can set it as a subset equality.
    \begin{equation}
        f(f^{-1}(B)) \subseteq B
    \end{equation}

    QED
\end{proof}

\subsection{Solution (b)}
\begin{equation}
    f(x) = \frac{1}{x}
\end{equation}

\pagebreak

\subsection{Solution (c)}
\begin{proof}
    We will first define the function $f$ and the function $f^{-1} \circ f$.
    \begin{gather}
        f(A) = \left\{f(x) | x \in A\right\}\\
        f^{-1}(f(A)) = \left\{f^{-1}(x) | x \in f(A)\right\}
    \end{gather}

    From this, we ca assume the existence of a generic element $x$ that is an element of $A$. 
    \begin{equation}
        x \in A
    \end{equation}

    Since $x \in A$, we can then apply the function $f$ to both sides.
    \begin{equation}
        f(x) \in f(A)
    \end{equation}

    We can then apply the definition of the image to this.
    \begin{equation}
        x \in f^{-1}(f(A))
    \end{equation}

    Since $x$ is a generic element of $A$, we have shown that every element of $A$ is in $f^{-1}(f(A))$.
    \begin{gather}
        A \subseteq f^{-1}(f(A))
    \end{gather}

    QED
\end{proof}

\subsection{Section (d)}
\[f(x) = \sqrt{x}\]

\pagebreak
\section{Exercise 1.7}
Suppose that $f$: $X \rightarrow Y$ and $g$: $Y \rightarrow X$  are functions and that the composite $g \circ f$ is the identity function  id: $X \rightarrow X$. (The identity function sends every element to itself:  id($x$) $= x$.) Show that $f$ must be a one-to-one function and that $g$ must be an onto function. 

\subsection*{Solution}
\newcommand{\id}{\text{id}}
First, we have to prove that $f$ must be a one-to-one function.
\begin{proof}
    First, we know that $g \circ f(x) = g(f(x)) = x$. 
    Suppose that we have two values $x, y \in X$ and that they have the same value when the function $f$ is applied to them.
    \begin{equation}
        f(x) = f(y)
    \end{equation} 
    We can use the function $g$ on both sides of this.
    \begin{equation}
        g(f(x)) = g(f(y))
    \end{equation}
    Since $g \circ f = \id$, we replace this.
    \begin{equation}
        \id(x) = \id(y)
    \end{equation}
    This is equivalent to our ending point.
    \begin{equation}
        x = y
    \end{equation}
    Putting it all together, we get an implication.
    \begin{equation}
        f(x) = f(y) \implies x = y
    \end{equation}
    Q.E.D.
\end{proof}

Now, we have to prove that $g$ is an $onto$ function.
As a reminder, a function $f: A \rightarrow B$ is onto (or surjective) if for every $b \in B$, there exists some $a \in A$ such that $f(a) = b$.
\begin{proof}[Proof by Contradiction]
    For contradiction, suppose that $\exists x \in X$ such that for $\forall y \in Y$, $g(y) \ne x$.
    The given information we have is that $g \circ f$ is the identity function $\id$.
    By its very nature, $\id$ is always bijective. 
    From that, we can establish that it applies to our value $x$, as does its equivalent.
    \begin{gather}
        \exists x \in X: x = \id(x)\\
        \exists x \in X: x = g(f(x))
    \end{gather}
    We can apply this to the given inequality.
    \begin{gather}
        g(y) \ne x\\
        g(y) \ne g(f(x))
    \end{gather}
    Since $f: X \to Y$ and $x \in X$, we can establish that $\exists y \in Y: y = f(x)$.
    \begin{equation}
        \exists y \in Y: f(x) = y
    \end{equation}
    We can take the final contradiction assumption, creating new values appropriately and apply substitution to it.
    Since the value of $y$ can be any value in $Y$, that value could be $f(x)$, which is established to exist.
    \begin{gather}
        \exists x \in X: g(y) \ne x\\
        \exists x \in X: g(f(x)) \ne x\\
        \exists x \in X: \id(x) \ne x
    \end{gather}
    This means that $x$ is not equal to its identity function.
    This contradicts the sheer fact of $\id(x) = x$.
    This shows that $\id(x) = x$ can only be true if our initial assumption is false.
    This proves our point by contradiction.\\
    Q.E.D.
\end{proof}


\pagebreak
\section{Exercise 1.8}
The following are special cases of De Morgan's laws.\\
(a) Prove that $(A \cap B)^c = A^c \cup B^c$. \\
(b) Prove that $(A \cup B)^c = A^c \cap B^c$. 

\subsection{Solution (a)}
Prove that $(A \cap B)^c = A^c \cup B^c$.
\begin{proof}[Proof by Truth Table]
    We will have all the following conditions apply for whether a given element $x$ is part of it.
    \begin{center}
        \begin{tabular}{| c | c | c | c |}
            \hline
            $A$ &$B$    &$A \cap B$ &$(A \cap B)^c$\\
            \hline
            T   &T  &T  &F\\
            T   &F  &F  &T\\
            F   &T  &F  &T\\
            F   &F  &F  &T\\
            \hline
        \end{tabular}
        \begin{tabular}{| c | c | c | c | c | c | c |}
            \hline
            $A$ &$B$    &$A^c$  &$B^c$  &$A^c \cup B^c$\\
            \hline
            T   &T  &F  &F  &F\\
            T   &F  &F  &T  &T\\
            F   &T  &T  &F  &T\\
            F   &F  &T  &T  &T\\
            \hline
        \end{tabular}
    \end{center}

    Since the columns of $(A \cap B)^c$ and $A^c \cup B^c$ are the same for the same values of $A$ and $B$ in their respective columns, we can conclude that $(A \cap B)^c = A^c \cup B^c$.\\
    Q.E.D.
\end{proof}

\subsection{Solution (b)}
Prove that $(A \cup B)^c = A^c \cap B^c$. 
\begin{proof}[Proof by Truth Table]
    We will have all the following conditions apply for whether a given element $x$ is part of it.
    \begin{center}
        \begin{tabular}{| c | c | c | c |}
            \hline
            $A$ &$B$    &$A \cup B$ &$(A \cup B)^c$\\
            \hline
            T   &T  &T  &F\\
            T   &F  &T  &F\\
            F   &T  &T  &F\\
            F   &F  &F  &T\\
            \hline
        \end{tabular}
        \begin{tabular}{| c | c | c | c | c | c | c |}
            \hline
            $A$ &$B$    &$A^c$  &$B^c$  &$A^c \cap B^c$\\
            \hline
            T   &T  &F  &F  &F\\
            T   &F  &F  &T  &F\\
            F   &T  &T  &F  &F\\
            F   &F  &T  &T  &T\\
            \hline
        \end{tabular}
    \end{center}

    Since the columns of $(A \cup B)^c$ and $A^c \cap B^c$ are the same for the same values of $A$ and $B$ in their respective columns, we can conclude that $(A \cup B)^c = A^c \cap B^c$.\\
    Q.E.D.
\end{proof}

\pagebreak
\section{Exercise 1.9}
(a) Prove that $\sqrt{3}$ is irrational. \\
(b) What goes wrong when you try to adapt your argument from  part (a) to show that $\sqrt{4}$ is irrational (which is absurd)? \\
(c) In part (a) you proved that $\sqrt{3}$ to be irrational, and essentially the same proof shows that $\sqrt{5}$ is irrational. By considering their product or otherwise, prove that $\sqrt{3} + \sqrt{5}$ and $\sqrt{3} - \sqrt{5}$ are either both rational or both irrational. Deduce that they must both be irrational. 

\subsection{Solution (a)}
Proof that $\sqrt{3}$ is irrational.
\begin{proof}[Proof by Contradiction]
    Suppose $\sqrt{3}$ is rational.
    This would mean that for some relatively prime $p, q \in \mathbb{Z}$, $\sqrt{3} = \frac{p}{q}$.
    Square both sides and solve for $p^2$.
    \begin{gather}
        \sqrt{3} = \frac{p}{q}\\
        3 = \frac{p^2}{q^2}\\
        p^2 = 3q^2
    \end{gather}

    By Euclid's Lemma, we know that if 3 divides $p^2$, then 3 divides $p$.
    As such, we can substitute in $p = 3j$ for some $j \in \mathbb{Z}$.
    We can continue with prior steps.
    \begin{gather}
        (3j)^2 = 3q^2\\
        3q^2 = 9j^2\\
        q^2 = 3j^2
    \end{gather}

    This tells us that 3 divides $q^2$ and by Euclid's Lemma that 3 divides $q$.
    This means that 3 divides both $p$ and $q$. 
    This means that $p$ and $q$ are not relatively prime, which contradicts our assumption.
    Therefore, $\sqrt{3}$ has to be irrational.\\
    Q.E.D.
\end{proof}

\subsection{Solution (b)}
\begin{proof}
    \begin{gather}
        \sqrt{4} = \frac{p}{q}\\
        4 = \frac{p^2}{q^2}\\
        p^2 = 4q^2
    \end{gather}
    
    Now, the issue we have here is that Euclid's lemma only works for prime numbers.
    Since 4 is not prime, we can only say that if 4 divides $p^2$, then $\sqrt{4} = 2$ divides p. 
    Here is what happens if we try to substitute in $p = 2j$ for sone $j \in \mathbb{Z}$.
    \begin{gather}
        (2j)^2  =   4q^2\\
        4j^2    =   4q^2\\
        j^2     =   q^2
    \end{gather}
    
    We can substitute this into our other equations. 
    However, we would be going in circles and would be unable to arrive at a contradiction.
    This proves that we cannot prove that $\sqrt{4}$ is irrational the way we did for $\sqrt{3}$.
\end{proof}

\subsection{Solution (c)}
First we prove they are both either rational or irrational.
\begin{proof}[Proof by Contradiction]
    We first consider the product of the two.
    \begin{align}
        (\sqrt{3} + \sqrt{5})(\sqrt{3} - \sqrt{5})
            &=  3 - 5
            =   -2
    \end{align}

    We can clearly see that the product of the two is a rational number. 
    Now, suppose (for contradiction purposes) that one and only one of two multiplied numbers is rational. 
    That number would be represented by a fraction of natural numbers $\frac{p}{q}$. 
    The other number would be irrational, and it would be represented be represented by $n$.
    \begin{gather}
        n*\frac{p}{q} = -2\\
        n * p = -2 * q\\
        n = \frac{-2*q}{p}
    \end{gather}

    Since $p$ and $q$ are both natural numbers, that means that $\frac{q}{p}$ is a rational number. 
    Since rationals are closed under multiplication, this means that $n$ is a rational number.
    This contradicts our initial assumption.
    This means that $\sqrt{3} + \sqrt{5} \in \mathbb{Q} \iff \sqrt{3} - \sqrt{5} \in \mathbb{Q}$.
\end{proof}

Now, we prove that neither are rational.
\begin{proof}[Proof by Contradiction]
    We will assume for contradiction purposes that both are rational.
    \begin{gather}
        \sqrt{3} + \sqrt{5} \in \mathbb{Q}\\
        \sqrt{3} - \sqrt{5} \in \mathbb{Q}
    \end{gather}

    Since rational numbers are closed under addition, we can add the two together and say that the result is rational.
    \begin{gather}
        (\sqrt{3} - \sqrt{5}) + (\sqrt{3} + \sqrt{5}) = 2\sqrt{3} \rightarrow
        2\sqrt{3} \in \mathbb{Q}
    \end{gather}

    Since 2 is rational and rationals are closed under division (except for 0), then we can say that our result divided by 2 is rational.
    \begin{gather}
        \frac{2\sqrt{3}}{2} = \sqrt{3} \rightarrow
        \sqrt{3} \in \mathbb{Q}
    \end{gather}

    This is a contradiction of an established statement. 
    This means that our initial statement is false. 
    Combining this with the fact that $\sqrt{3} + \sqrt{5} \in \mathbb{Q} \iff \sqrt{3} - \sqrt{5} \in \mathbb{Q}$, we can say that $\sqrt{3} + \sqrt{5} \notin \mathbb{Q}$ and $\sqrt{3} - \sqrt{5} \notin \mathbb{Q}$.\\
    QED
\end{proof}

\pagebreak
\section{Problem 1.10}
Prove that the multiplicative identity in a field is unique. 

\subsection*{Solution}
We'll prove this by contradiction.
\begin{proof}[Proof by Contradiction]
    Suppose that the field has two multiplicative identity elements. 
    The first is 1.
    Suppose that we call the second one $h$ and that $h \neq 1$.
    For $h$ to be a multiplicative identity, it would mean that $h \in \mathbb{F}$ and for all $a \in \mathbb{F}$, $a \cdot h = a$.
    Suppose we were to multiply our two multiplicative identities together.
    Due to its nature covered in the identity law, $h \cdot 1 = h$.
    However, our earlier statement would lead us to believe that $h \cdot 1 = 1$.
    This would only be possible if $h = 1$.
    This would give us a contradiction, which would mean that the multiplicative identity would be unique.
    QED
\end{proof}

\pagebreak
\section{Problem 1.11}
Given an ordered field $\mathbb{F}$, recall that we defined the positive elements to be a nonempty subset $P \subseteq \mathbb{F}$ that satisfies both of the following conditions:

(i) If $a, b \in P$, then $a + b \in P$ and $a \cdot b \in P$;  

(ii) If $a \in \mathbb{F}$ and $a \ne 0$, then either $a \in P$ or $-a \in P$, but not  both. \\
(a) Give an example of some $P_1 \subseteq \mathbb{R}$ that satisfies (i) but not (ii).\\
(b) Give an example of some $P_2 \subseteq \mathbb{R}$ that satisfies (ii) but not (i). 

\subsection{Solution (a)}
For the set of integers in the range $[-\infty,\infty]$, (i) is satisfied, but not (ii). 

\subsection{Solution (b)}
For the set of integers in the range $[1,2]$, (ii) is satisfied, but not (i). 


\pagebreak
\section{Exercise 1.12}
Assume that $\mathbb{F}$ is an ordered field and $a, b, c, d \in \mathbb{F}$ with $a<b$ and $c<d$. \\
(a) Show that $a + c < b + d$. \\
(b) Prove that it is not necessarily true that $ac < bd$. \\
Note whenever you use an axiom. 

\subsection{Solution (a)}
\begin{proof}
    We know that if $a<b$ and $c<d$, then $b - a \in P$ and $d - c \in P$.
    By the order axiom, 
    \begin{gather}
        (b - a) + (d - c) \in P\\
        (b + d) - (a + c) \in P
    \end{gather}

    By the definition of the $<$ operator,
    \begin{gather}
        a + c < b + d
    \end{gather}
    QED
\end{proof}

\subsection{Solution (b)}
Counterexample:
\begin{gather}
    a = -1; b = 1 \implies a < b\\
    c = -2; d = 2 \implies c < d\\
    a * c = 2; b * d = 2 \implies ac = bd
\end{gather}
This satisfies the case of a counterexample.


\pagebreak
\section{Exercise 1.13}
Let $a, b$ and $\varepsilon$ be elements of an ordered field. \\
(a) Show that if $a < b + \varepsilon$ for every $\varepsilon > 0$, then $a \le b$. \\
(b) Use part (a) to show that if $|a - b| < \varepsilon$ for all $\varepsilon > 0$, then $a = b$. \\
Note whenever you use an axiom. 

\subsection{Solution (a)}
\begin{proof}[Proof by Contradiction]
    Suppose that $b < a$.
    We also know that $\varepsilon > 0$.
    These both would themselves mean that $a - b \in P$ and $\varepsilon \in P$.
    We can put the two of these together by the order axiom, as well as establish some other things.
    \begin{gather}
        a - b + \varepsilon \in P\\
        a - b > 0
    \end{gather}
    It could be that $\varepsilon = \frac{a - b}{2}$.
    Since $\varepsilon > 0$, this is applicable.
    Our given information tells us about further relationships.
    \begin{gather}
        a < b + \varepsilon
    \end{gather}
    By the order axiom.
    \begin{gather}
        b + \varepsilon - a \in P
    \end{gather}
    By the associative law.
    \begin{gather}
        \varepsilon - (a - b) \in P
    \end{gather}
    By the order axiom.
    \begin{gather}
        a - b < \varepsilon
    \end{gather}
    By an equivalency.
    \begin{gather}
        a - b < \frac{a - b}{2}
    \end{gather}
    This is a contradiction of itself.\\
    QED
\end{proof}

\subsection{Solution (b)}
\begin{proof}
    We have a known equation.
    \begin{equation}
        |a - b| < \varepsilon
    \end{equation}

    By Note 1.11.g.
    \begin{equation}
        -\varepsilon < a - b < \varepsilon
    \end{equation}

    This gets split into two equations.
    \begin{gather}
        a - b < \varepsilon\\
        -\varepsilon < a - b
    \end{gather}

    By Note 1.9.a, adding either $b$ or $b + \varepsilon$, respectively.
    \begin{gather}
        a < b + \varepsilon\\
        b < a + \varepsilon
    \end{gather}

    We can apply part (a).
    \begin{gather}
        a \le b\\
        b \le a
    \end{gather}

    There is only case where this does not self-contradict.
    \begin{equation}
        a = b
    \end{equation}
    QED
\end{proof}


\pagebreak
\section{Exercise 1.14}
Prove that the equality $\left|ab\right| = \left|a\right|\cdot\left|b\right|$ holds for  all real numbers a and b. 

\subsection*{Solution}
\begin{proof}[Proof by Cases]
    \underline{Case 1: $a, b \in P$}
    \begin{gather}
        a = |a|\\
        b = |b|\\
        \left|ab\right| = a * b = ab\\
        \left|a\right| \cdot \left|b\right| = a \cdot b = ab
    \end{gather}
    As desired.

    \underline{Case 2: $a \in P, b \notin P$}
    This applies equally for $a \notin P, b \in P$.
    \begin{gather}
        |a| = a\\
        |b| = -b\\
        |a|*|b| = a * (-b) = -ab
    \end{gather}

    Since $a > 0$ and $b < 0$:
    \begin{gather}
        ab < 0\\
        |ab| = -ab
    \end{gather}
    As desired.

    \underline{Case 3: $a, b \notin P$}
    \begin{gather}
        |a| = -a\\
        |b| = -b\\
        |a||b| = ab\\
    \end{gather}

    Since $a < 0$ and $b < 0$:
    \begin{gather}
        ab > 0\\
        |ab| = ab
    \end{gather}
    As desired.

    QED
\end{proof}

\pagebreak
\section{Exercise 1.15}
For each of the following, find all numbers $x$ which satisfy the expression. 

\begin{multicols}{2}
    \begin{itemize}
        \item[(a)] $\left|x - 4\right| = 7$
        \item[(b)] $\left|x - 4\right| < 7$
        \item[(c)] $\left|x + 2\right| < 7$
        \item[(d)] $\left|x - 1\right| + \left|x - 2\right| > 1$
        \item[(e)] $\left|x - 1\right| + \left|x + 1\right| > 1$
        \item[(f)] $\left|x - 4\right| > 1$
        \item[(g)] $\left|x - 1\right| \cdot \left|x + 1\right| = 0$
        \item[(h)] $\left|x - 1\right| \cdot \left|x + 2\right| = 3$
    \end{itemize}
\end{multicols}

\subsection{Solution (a) x = 4 and x = -3}
Case 1: $|x - 4| = x - 4$
\begin{gather}
    x - 4 = 7\\
    x = 11
\end{gather}

Case 2: $|x - 4| = -(x - 4)$
\begin{gather}
    -(x - 4) = 7\\
    4 - x = 7\\
    x = -3
\end{gather}

\subsection{Solution (b) $x \in (-3,7)$}
Case 1: $|x - 4| = x - 4$
\begin{gather}
    x - 4 < 7\\
    x < 11
    x \in (-\infty,7)
\end{gather}

Case 2: $|x - 4| = -(x - 4)$
\begin{gather}
    -(x - 4) < 7\\
    x - 4 > -7\\
    x > -3
    x \in (-3,\infty)
\end{gather}

Combination:
\begin{align}
    x   &\in (-\infty,7) \andtxt x \in (-3,\infty)\\
    x   &\in (-\infty,7) \cap (-3,\infty)\\
    x   &\in (-3,7)
\end{align}

\subsection{Solution (c) $-9 < x < 5$}
There is one point where an absolute value would change at $x = -2$ that divides the field in two.

Case 1: $x < -2$\\
In this instance, $|x + 2| = -(x + 2) = -x - 2$.
\begin{gather}
    |x + 2| < 7\\
    -x - 2 < 7\\
    -9 < x
\end{gather}

Case 2: $x > -2$\\
In this instance, $|x + 2| = x + 2$
\begin{gather}
    |x + 2| < 7\\
    x + 2 < 7\\
    x < 5
\end{gather}

Consolidation\\
Putting our two cases together, we get \boxed{-9 < x < 5}.

\subsection{Solution (d) $x \in (-\infty, -2) \andtxt x \in (2, \infty)$}
There are two points where the absolute values would change, at $x = 1$ and $x = 2$, which divides the field in three.
We can consider all three of these cases.

Case 1: $x < 1$\\
In this instance, $|x - 1| = -(x - 1) = 1 - x$ and $|x - 2| = -(x - 2) = 2 - x$.
\begin{gather}
    \left|x - 1\right| + \left|x - 2\right| > 1\\
    1 - x + 2 - x > 1\\
    3 - 2x > 1\\
    4 > 2x\\
    x < 2
\end{gather}

Case 2: $1 < x < 2$\\
In this instance, $|x - 1| = x - 1$ and $|x - 2| = -(x - 2) = 2 - x$.
\begin{gather}
    \left|x - 1\right| + \left|x - 2\right| > 1\\
    x - 1 + 2 - x > 1\\
    1 > 1
\end{gather}
This is false, so it cannot be the case.

Case 3: $x > 2$\\
In this instance, $|x - 1| = x - 1$ and $|x - 2| = x - 2$.
\begin{gather}
    \left|x - 1\right| + \left|x - 2\right| > 1\\
    x - 1 + x - 2 > 1\\
    2x - 3 > 1\\
    2x > 4\\
    x > 2
\end{gather}

Consolidation\\
This gives a final range for x.
\begin{equation}
    \boxed{x \in (-\infty, -2) \andtxt x \in (2, \infty)}
\end{equation}

\pagebreak
\section{Exercise 1.16}
Let $\max\{x, y\}$ denote the maximum of the real numbers $x$ and $y$, and let $\min\{x, y\}$ denote the minimum. 
For example, $\min\{-1, 4\} = -1$, and also $\min\{-1, -1\} = -1$. 
Prove that
\begin{gather*}
    \max\{x,y\} = \frac{x + y + \left|y - x\right|}{2}
    \andtxt
    \min\{x,y\} = \frac{x + y - \left|y - x\right|}{2}
\end{gather*}

Then, find the formula for $\max\{x,y,z\}$ and $\min\{x,y,z\}$.

\subsection*{Solution}
\begin{proof}[Proof by Cases]
    There are three cases to cover: $x > y$, $x < y$, and $x = y$. 

    \underline{Case 1}: $x > y$\\
    This would mean that $|y - x| = -(y - x) = x - y$.
    Our goal case would be that $\max\{x,y\} = x$ and that $\min\{x,y\} = y$.
    \begin{align}
        \max\{x,y\} &=  \frac{x + y + |y - x|}{2}
            =   \frac{x + y + x - y}{2}
            =   \frac{2x + 0y}{2}
            =   x
    \end{align}
    For maximum, this works.
    \begin{equation}
        \min\{x,y\} =   \frac{x + y - |y - x|}{2}
            =   \frac{x + y + y - x}{2}
            =   \frac{0x + 2y}{2}
            =   y
    \end{equation}
    For minimum, this works. 
    Conclusively, Case 1 works for both minimum and maximum.

    \underline{Case 2}: $x < y$\\
    This would mean that $|y - x| = y - x$.
    Our goal case would be that $\max\{x,y\} = x$ and that $\min\{x,y\} = y$.
    \begin{align}
        \max\{x,y\} &=  \frac{x + y + |y - x|}{2}
            =   \frac{x + y + y - x}{2}
            =   \frac{0x + 2y}{2}
            =   y
    \end{align}
    For maximum, this works.
    \begin{equation}
        \min\{x,y\} =   \frac{x + y - y + x}{2}
            =   \frac{x + y - y + x}{2}
            =   \frac{2x + 0y}{2}
            =   x
    \end{equation}
    For minimum, this works. 
    Conclusively, Case 2 works for both minimum and maximum.

    \underline{Case 3}: $x = y$\\
    This would mean that $|y - x| = 0$.
    Our goal would be that the maximum and minimum would be either of the results with substitution and neither without.
    \begin{align}
        \max\{x,y\} &=  \frac{x + y + |y - x|}{2}
            =   \frac{x + y}{2}\\
        \max\{x,y\} &=  \frac{2x}{2}
            =   x\\
        \max\{x,y\} &=  \frac{2y}{2}
            =   y
    \end{align}
    For maximum, this works.
    \begin{align}
        \min\{x,y\} &=  \frac{x + y - |y - x|}{2}
            =   \frac{x + y}{2}\\
        \min\{x,y\} &=  \frac{2x}{2}
            =   x\\
        \min\{x,y\} &=  \frac{2y}{2}
            =   y
    \end{align}
    For minimum, this works.
    Conclusively, Case 3 works for both maximum and minimum.

    Q.E.D.
\end{proof}

For finding the maximum and minimum values of three numbers, we can know that $\max\{x,y,z\} = \max\{\max\{x,y\},z\}$ and that $\min\{x,y,z\} = \min\{\min\{x,y\},z\}$.
\begin{align}
    \max\{x,y,z\}   &=  \max\{\max\{x,y\},z\}\\
        &=  \frac{\max\{x,y\} + z + \left| z - \max\{x,y\} \right|}{2}\\
        &=  \frac{\frac{x + y + |y - x|}{2} + z + \left| z - \frac{x + y + |y - x|}{2} \right|}{2}\\
        &=  \frac{x + y + |y - x| + 2z + 2\left| z - \frac{x + y + |y - x|}{2} \right|}{2}\\
        &=  \boxed{\frac{x + y + |y - x| + 2z + \left| 2z - x - y - |y - x| \right|}{4}}
\end{align}

\pagebreak
Case 1: $|y - x| = y - x$
\begin{align}
    \max\{x,y,z\}   &=  \frac{x + y + y - x + 2z + \left| 2z - x - y - y + x \right|}{4}\\
        &=  \frac{2y + 2z + \left| 2z - 2y \right|}{4}\\
        &=  \frac{y + z + \left| z - y \right|}{2}
        =   \max\{y,z\}
\end{align}

Case 2: $|y - x| = -(y - x) = x - y$
\begin{align}
    \max\{x,y,z\}   &=  \frac{x + y + x - y + 2z + \left| 2z - x - y - x + y \right|}{4}\\
        &=  \frac{2x + 2z + \left| 2z - 2x \right|}{4}\\
        &=  \frac{x + z + \left| z - x \right|}{2}
        =   \max\{x,z\}
\end{align}

$\min\{x,y,z\}$ can be found in a similar way.

\pagebreak
\section{Exercise 1.17}
Prove that if $a, b \in R$ and $0 < a < b$, then  $a^n < b^n$ for any positive integer n. 

\subsection*{Solution}
\begin{proof}[Proof by Induction]
    Induction comes in two parts, firstly from the smallest value. 
    The smallest value here would be $n = 1$.
    \begin{gather}
        a^n < b^n\\
        a^1 < b^1\\
        a < b
    \end{gather}
    This is a given and confirms the base case requirement.

    Next, we have to prove that $a^n < b^n$ implies $a^{n + 1} < b^{n + 1}$.
    We can reduce the value on one of the sides to 1.
    \begin{gather}
        a^n < b^n\\
        a^n (a^n)^{-1} < (a^n)^{-1} b^n\\
        1 < (a^n)^{-1} b^n
    \end{gather}

    We can do the same with $a < b$, our base case and given.
    \begin{gather}
        a < b\\
        a \cdot b^{-1} < b \cdot b^{-1}\\
        a \cdot b^{-1} < 1
    \end{gather}

    The transistive property can be applied here.
    \begin{gather}
        a \cdot b^{-1} < (a^n)^{-1} b^n\\
        a^n a \cdot b^{-1} \cdot b < a^n (a^n)^{-1} b^n \cdot b\\
        a^n \cdot a < b^n \cdot b\\
        a^{n + 1} < b^{n + 1}
    \end{gather}
    This confirms the successive requirement.
    As a whole, this confirms the statement.

    Q.E.D.
\end{proof}

\pagebreak
\section{Exercise 1.18}
Prove that if $a_1, a_2,\dots,a_n$ are real numbers, then 
\begin{gather*}
    \left| a_1 + a_2 + \dots + a_n \right| \le \left|a_1\right| + \left|a_2\right| + \dots + \left|a_n\right|
\end{gather*}

\subsection*{Solution}
\begin{proof}[Proof by Induction]
    Before we begin, we can rewrite our target equation.
    \begin{gather}
        \left| a_1 + a_2 + \dots + a_n \right| \le \left|a_1\right| + \left|a_2\right| + \dots + \left|a_n\right|\\
        \left|\sum_{i = 1}^{n} a_i\right| \leq \sum_{i = 1}^{n} \left| a_i \right|
    \end{gather}

    Case 1: Minimum value of $n = 1$\\
    We have a value for what would be the left side of the equation.
    \begin{gather}
        \left| a_1 \right|
    \end{gather}

    By the reflextive property, we can create equation for it.
    \begin{equation}
        \left| a_1 \right| = \left| a_1 \right|
    \end{equation}

    This stands in concurrence with the requirement of the base case.

    Case 2: $n \implies n + 1$\\
    We can start with the value of the function for n.
    \begin{gather}
        \left| a_1 + a_2 + \dots + a_n \right| \le \left|a_1\right| + \left|a_2\right| + \dots + \left|a_n\right|\\
        \left|\sum_{i = 1}^{n} a_i\right| \leq \sum_{i = 1}^{n} \left| a_i \right|
    \end{gather}

    We can add $\left| a_{n + 1} \right|$ to both sides.
    \begin{gather}
        \left|\sum_{i = 1}^{n} a_i\right| + \left| a_{n + 1} \right| \leq \sum_{i = 1}^{n} \left| a_i \right| + \left| a_{n + 1} \right|
    \end{gather}

    We can take the left side and apply the triangle inequality to it.
    \begin{equation}
        \left|\sum_{i = 1}^{n} a_i + a_{n + 1} \right| \leq \left|\sum_{i = 1}^{n} a_i\right| + \left| a_{n + 1} \right|
    \end{equation}

    We can then use the transistive property.
    \begin{gather}
        \left|\sum_{i = 1}^{n} a_i + a_{n + 1} \right| \leq \sum_{i = 1}^{n} \left| a_i \right| + \left| a_{n + 1} \right|
    \end{gather}

    Put the like terms into sums.
    \begin{gather}
        \left|\sum_{i = 1}^{n + 1} a_i \right| \leq \sum_{i = 1}^{n + 1} \left| a_i \right|
    \end{gather}

    This can be rewritten.
    \begin{equation}
        \left| a_1 + a_2 + \dots + a_n + a_{n + 1} \right| \le \left|a_1\right| + \left|a_2\right| + \dots + \left|a_n\right| + \left|a_{n + 1}\right|
    \end{equation}

    This is indeed our target to prove by induction.
    Q.E.D.
\end{proof}

\pagebreak
\section{Exercise 1.19}
Prove that $\overunderset{n}{k = 1}{\Sigma}\frac{1}{k(k + 1)} = \frac{n}{n + 1}$ for every natural number n. 

\subsection*{Solution}
\begin{proof}[Proof by Induction]
    This is a proof by induction, so we will prove a base case, then prove that every successive number's success implies it works for the next.

    Case 1: Base case $n = 1$
    \begin{gather}
        \sum^{n}_{k = 1}\frac{1}{k(k + 1)} = \frac{n}{n + 1}\\
        \sum^{1}_{k = 1}\frac{1}{k(k + 1)} = \frac{1}{1 + 1}\\
        \frac{1}{1(1 + 1)} = \frac{1}{1 + 1}\\
        \frac{1}{2} = \frac{1}{2}
    \end{gather}
    The base case works.

    Case 2: $n \implies n + 1$
    \begin{gather}
        \sum^{n}_{k = 1}\frac{1}{k(k + 1)} = \frac{n}{n + 1}
    \end{gather}

    We can add the value of $\frac{1}{(n + 1)((n + 1) + 1)}$ to both sides.
    \begin{gather}
        \sum^{n}_{k = 1}\frac{1}{k(k + 1)} + \frac{1}{(n + 1)((n + 1) + 1)} = \frac{n}{n + 1} + \frac{1}{(n + 1)((n + 1) + 1)}\\
        \sum^{n + 1}_{k = 1}\frac{1}{k(k + 1)} = \frac{n}{n + 1} + \frac{1}{(n + 1)(n + 2)}
    \end{gather}

    We can now focus on the term on the right side.
    \begin{gather}
        \frac{n}{n + 1} + \frac{1}{(n + 1)(n + 2)}\\
        \frac{1}{n + 1}\left( n + \frac{1}{n + 2} \right)\\
        \frac{1}{n + 1}\left( \frac{n(n + 2) + 1}{n + 2} \right)\\
        \frac{n^2 + 2n + 1}{(n + 1)(n + 2)}\\
        \frac{(n + 1)^2}{(n + 1)(n + 2)}\\
        \frac{n + 1}{n + 2}\\
        \frac{n + 1}{(n + 1) + 1}
    \end{gather}

    As such, we can get a final equation.
    \begin{equation}
        \sum^{n + 1}_{k = 1}\frac{1}{k(k + 1)} = \frac{n + 1}{(n + 1) + 1}
    \end{equation}

    This is the value we were looking to get, proving the statement by (weak) induction.

    Q.E.D.
\end{proof}

\pagebreak
\section{Exercise 1.20}
Determine which natural numbers, $n$, have the  property that $\sqrt{n}$ is irrational. 

\pagebreak
\section{Exercise 1.37}
(a) Determine
\[ \left\{1,3,5\right\} \cdot \left\{-3,0,1\right\} \]
(b) Give an example of sets $A$ and $B$ where $\sup(A\cdot B) \ne \sup(A) \cdot \sup(B)$.

\chapter{Theories of Cantor (Somehow not Jewish)}
\section{Exercise 2.1}
\begin{itemize}{}
    \item (a) List all the elements of $\mathcal{P}(\{a,b,c\})$. 
    \item (b) Define a formula for the number of elements in the power set of an n-element set.
\end{itemize}

\subsection{Solution (a)}
The power set is the set of all subsets.
Below is a list of all such subsets.
\begin{center}
    \begin{tabular}{|c|c|c|}
        \hline
        1 & 2 & 3 \\
        \hline
        \{a\} & \{b\} & \{c\} \\
        \hline \hline
        4 & 5 & 6 \\
        \hline
        \{a,b\} & \{a,c\} & \{b,c\} \\
        \hline \hline
        7 & 8 & \\
        \hline
        \{a,b,c\} & $\emptyset$ & \\
        \hline
    \end{tabular}
\end{center}

There are \boxed{8} members of $\mathcal{P}(\{a,b,c\})$. 

\subsection{Solution (b)}
We can look at the first couple values.
\begin{center}
    \begin{tabular}{|c|c|c|c|c|c}
        \hline
        n & 0 & 1 & 2 & 3 \\
        \hline
        $\mathcal{P}$ & 1 & 2 & 4 & 8 \\
        \hline
    \end{tabular}
\end{center}

As it stands, it appears that the proper formula for it is \boxed{2^n}. 

\pagebreak
\section{Exercise 2.2}
Prove that $\left|\left\{ e^n : n \in \mathbb{N} \right\}\right| = \left| \mathbb{N} \right|$.

\subsection*{Solution}
\begin{proof}
    Let there be a function $f$.
    \begin{gather}
        f: \mathbb{N} \rightarrow e^\mathbb{N}\\
        n \rightarrowtail e^n
    \end{gather}
    
    This pairs every natural number with a number in the left set.
    Conclusively, the two sets have the same cardinality and there is a bijection.
    THis can be written in math terms.
    \begin{equation}
        \left|\left\{ e^n : n \in \mathbb{N} \right\}\right| = \left| \mathbb{N} \right|
    \end{equation}
    Q.E.D.
\end{proof}

\pagebreak
\section{Exercise 2.3}
The following pairs of sets have the same size, and  so there exists a bijection between them. Write down an explicit  bijection in each case. You do not need to prove your answers.  
\begin{multicols}{2}
    (a) $(0, \infty)$ and $(1, \infty)$ 

    (b) $(0, \infty)$ and $(-\infty, 3)$

    (c) $(0, \infty)$ and (0, 1) 

    (d) $\mathbb{R}$ and $(0, \infty)$  
    \columnbreak

    (e) $\mathbb{R}$ and (0, 1) 
    
    (f) $\mathbb{Z}$ and $\{ ..., \frac{1}{8}, \frac{1}{4}, \frac{1}{2}, 1, 2, 4, 8, ... \}$  
    
    (g) $\{0, 1\} \times \mathbb{N}$ and $\mathbb{N}$  
    
    (h) [0, 1] and (0, 1) 
\end{multicols}

\subsection{Solution (a)}
\begin{multicols}{2}
    \begin{gather}
        f : (0,\infty) \to (1,\infty)\\
        n \rightarrowtail n + 1
    \end{gather}
    
    \columnbreak
    \begin{gather}
        f : (1,\infty) \to (0,\infty)\\
        n \rightarrowtail n - 1
    \end{gather}
\end{multicols}

\subsection{Solution (b)}
\begin{multicols}{2}
    \begin{gather}
        f : (0,\infty) \to (-\infty, 3)\\
        n \rightarrowtail -n + 3
    \end{gather}

    \columnbreak
    \begin{gather}
        f : (-\infty, 3) \to (0,\infty)\\
        n \rightarrowtail -n + 3
    \end{gather}
    
\end{multicols}

\subsection{Solution (c)}
\begin{multicols}{2}
    \begin{gather}
        f : (0,\infty) \to (0,1)\\
        n \rightarrowtail \frac{1}{n + 1}
    \end{gather}

    \columnbreak
    \begin{gather}
        f : (0,1) \to (0,\infty)\\
        n \rightarrowtail \frac{1}{n} - 1
    \end{gather}
    
\end{multicols}

\subsection{Solution (d)}
\begin{multicols}{2}
    \begin{gather}
        f : \mathbb{R} \to (0, \infty)\\
        n \rightarrowtail e^x
    \end{gather}

    \columnbreak
    \begin{gather}
        f : (0, \infty) \to \mathbb{R}\\
        n \rightarrowtail \ln(x)
    \end{gather}
\end{multicols}

\subsection{Solution (e)}
\begin{multicols}{2}
    \begin{gather}
        f : \mathbb{R} \to (0, 1)\\
        n \rightarrowtail \frac{1}{2n - 1}
    \end{gather}

    \columnbreak
    \begin{gather}
        f : (0, 1) \to \mathbb{R}\\
        n \rightarrowtail \frac{1}{2n} + \frac{1}{2}
    \end{gather}
\end{multicols}

\subsection{Solution (f)}
\begin{multicols}{2}
    \begin{gather}
        f : \mathbb{Z} \to \{ ..., \frac{1}{8}, \frac{1}{4}, \frac{1}{2}, 1, 2, 4, 8, ... \}\\
        n \rightarrowtail 2^n
    \end{gather}

    \columnbreak
    \begin{gather}
        f : \{ ..., \frac{1}{8}, \frac{1}{4}, \frac{1}{2}, 1, 2, 4, 8, ... \} \to \mathbb{Z}\\
        n \rightarrowtail \log_2 (n)
    \end{gather}
\end{multicols}

\subsection{Solution (g)}
\begin{multicols}{2}
    \begin{gather}
        f : \{0, 1\} \times \mathbb{N} \to \mathbb{N}\\
        (a,b) \rightarrowtail a \cdot b
    \end{gather}
    \columnbreak
    \begin{gather}
        f : \mathbb{N} \to \{0, 1\} \times \mathbb{N}\\
        n \rightarrowtail \left(n \mod 2, \frac{n - (n \mod 2)}{2}\right)
    \end{gather}
\end{multicols}

\subsection{Solution (h)}
Iffy solution. 
\begin{multicols}{2}
    \begin{gather}
        f : [0, 1] \to (0, 1)\\
        n \rightarrowtail \begin{cases}
            \frac{1}{2} & \text{if } n=0\\
            \frac{1}{2^{x+2}} & \text{if } n=\frac{1}{2^x}\\
            n    &   \text{otherwise}
        \end{cases}
    \end{gather}

    \columnbreak
    \begin{gather}
        f : (0, 1) \to [0,1]\\
        n \rightarrowtail \begin{cases}
            0 & \text{if } n=\frac{1}{2}\\
            \frac{1}{2^{x-2}} & \text{if } n=\frac{1}{2^x}\\
            n    &   \text{otherwise}
        \end{cases}
    \end{gather}
\end{multicols}

\pagebreak
\section{Exercise 2.4}
This problem shows that ``equinumerosity is an equivalence relation.'' (This justifies the notation $|A| = |B|$.) Let $A$, $B$, and $C$ be sets. For this problem only, we'll write $A \sim B$ to mean that $A$ and $B$ are equinumerous, meaning that there is a bijection $A \to B$. 

(a) Show that $A \sim A$.  

(b) Show that if $A \sim B$ then $B \sim A$.  

(c) Show that if $A \sim B$ and $B \sim C$, then $A \sim C$. 

\subsection{Solution (a)}
We can always establish a bijection between $A$ and $A$.
\begin{gather}
    f: A \to A \\
    n \rightarrowtail n
\end{gather}
This establishes the bijection. 

\subsection{Solution (b)}
It is given that $A \sim B$. 
We can extract from that that the function from $A$ to $B$ is bijective, and consequently that it is reversable.
Said reversability can give us the function from $B$ to $A$, which would itself be bijective. 
This bijectivity says that $B \sim A$. 

\subsection{Solution (c)}
Since $A \sim B$, then $A \to B$ is bijective.
The same can be done to conclude that $B \to C$ is bijective.
Since both are bijective, this means that $f(a) = f(b) \implies a = b$ and $f(b) = f(c) \implies b = c$, as well as that $b \in B \implies \exists a \in A : f(a) = b$ and $c \in C \implies \exists b \in B : f(b) = c$.
By transistivity, since both $A \to B$ and $B \to C$ are bijective, then $A \to C$ is bijective.
This in turn means that $A \sim C$.

\pagebreak
\section{Exercise 2.5}
\begin{itemize}
    \item (a) Prove that if A and B are countable sets, then $A \cup B$ is also a countable set.  
    
    \item (b) Prove that if $A_n$ is a countable set for each $n \in N$, then the set  $\overunderset{\infty}{n = 1}{\bigcup} A_n$ is also countable. 
\end{itemize}

\subsection{Solution (a)}
We can establish a bijective function. 
\begin{multicols}{2}
    \begin{gather}
        f: \mathbb{N} \to A \cup B\\
        n \to \begin{cases}
            A_{\lfloor\frac{x}{2}\rfloor} & \text{if } x \equiv 0 (\mod 2)\\
            B_{\lfloor\frac{x}{2}\rfloor} & \text{if } x \equiv 1 (\mod 2)
        \end{cases}
    \end{gather}
    
    \columnbreak
    \begin{gather}
        f: A \cup B \to \mathbb{N}\\
        (A \cup B)_n \to n
    \end{gather}
\end{multicols}

This bijection establishes that $A \cup B$ is countable.

\subsection{Solution (b)}
This can be done using the winding bijection, detailed in the textbook.
The table for this is below.
\begin{center}
    \begin{tabular}{c| c c c c c}
            &1  &2  &3  &4  &\dots\\
        \hline
        $A_1$ &$a_{11}$    &$a_{21}$    &$a_{31}$   &$a_{41}$   &\dots\\
        $A_2$ &$a_{12}$    &$a_{22}$    &$a_{32}$   &$a_{42}$   &\dots\\
        $A_3$ &$a_{13}$    &$a_{23}$    &$a_{33}$   &$a_{43}$   &\dots\\
        $A_4$ &$a_{14}$    &$a_{24}$    &$a_{34}$   &$a_{44}$   &\dots\\
        \vdots  &\vdots    &\vdots      &\vdots     &\vdots     &$\ddots$
    \end{tabular}
\end{center}
This visual proof establishes the solution.

\pagebreak
\section{Exercise 2.6}
Show that $|\mathbb{N}| = |\mathbb{Z}|$ by finding an explicit bijection  from $\mathbb{N}$ to $\mathbb{Z}$. You do not need to prove your bijection works. 

\subsection*{Solution}
\begin{gather}
    f: \mathbb{N \to Z}\\
    n \to \lfloor\frac{n}{2}\rfloor (-1)^n
\end{gather}

\pagebreak
\section{Exercise 2.7}
Let $A, B \subseteq \mathbb{R}$. We define  $A \cdot B = \{a \cdot b: a \in A \andtxt b \in B\}$.  

(a) Give an example of sets $A_1$ and $B_1$ where $\left|A_1 \cdot B_1\right| < \max\{|A_1|, |B_1|\}$.  

(b) Give an example of sets $A_2$ and $B_2$ where $\left|A_2 \cdot B_2\right| > \max\{|A_2|, |B_2|\}$.

(c) Give an example of sets $A_3$ and $B_3$ where $\left|A_3 \cdot B_3\right| = \max\{|A_3|, |B_3|\}$.

(d) For which of the above does there exist an example where one or both of the sets are infinite? 

\subsection{Solution (a)}
\begin{gather}
    A_1 = \mathbb{N}\\
    B_1 = \{0\}\\
    A_1 \cdot B_1 = \{0\}
\end{gather}

\subsection{Solution (b)}
\begin{gather}
    A_2 = \{n : \mathbb{N} \andtxt 0 < n < 100\}\\
    B_2 = \{-1, 1\}\\
    A_2 \cdot B_2 = \{-99, -98, \dots , -1, 1, 2, \dots , 98, 99\}
\end{gather}

\subsection{Solution (c)}
\begin{gather}
    A_3 = \{1, 2, 3, 4, 5\}\\
    B_3 = \{1\}\\
    A_3 \cdot B_3 = \{1,2,3,4,5\}
\end{gather}

\subsection{Solution (d)}
For (a) and (c), yes.

\pagebreak
\section{Exercise 2.8}
(a) Describe a way to partition the set $\mathbb{N}$ into 6 subsets, each  containing infinitely many elements.\\ 
(b) Describe a way to partition the set $\mathbb{N}$ into infinitely many subsets, each containing infinitely many elements.

\subsection{Solution (a)}
Divide each number by 6 and take the remainder. 
The value of each remainder determines the set it would go in. The way to define each set is as follows:
\begin{align}
    &f: \mathbb{N} \to N_1  &n \to (n-1)*6 + 1\\
    &f: \mathbb{N} \to N_2  &n \to (n-1)*6 + 2\\
    &f: \mathbb{N} \to N_3  &n \to (n-1)*6 + 3\\
    &f: \mathbb{N} \to N_4  &n \to (n-1)*6 + 4\\
    &f: \mathbb{N} \to N_5  &n \to (n-1)*6 + 5\\
    &f: \mathbb{N} \to N_6  &n \to (n-1)*6 + 6
\end{align}

\subsection{Solution (b)}
Create one set $S_1$ and add one object: 1.
Create a second set $S_2$. 
Add 2 to $S_1$ and 3 to $S_2$.
Create $S_3$.
Add 4 to $S_1$, 5 to $S_2$, and 6 to $S_3$.
Your sets should look like the below.
\begin{align}
    S_1 &=  \{1,2,4\}\\
    S_2 &=  \{3,5\}\\
    S_3 &=  \{6\}
\end{align}

A pattern should be discernable here. 
We can define each individual set.
\begin{gather}
    f: \mathbb{N} \to S_i\\
    n \to \overunderset{i}{x = 0}{\Sigma} (x) + \overunderset{i + n - 1}{x = i}{\Sigma} (x) 
\end{gather}

\pagebreak
\section{Exercise 2.9}
Is $|\mathbb{Z \times N}|$ countable or uncountable? 

\subsection*{Solution}
Previously mentioned is the digonalization and pairing up practice, which has established $\mathbb{Q}$ as being countably infinite. 
In this instance, we can do the same, but instead of fractions, we can put $\mathbb{N}$ on one side and $\mathbb{Z}$ on the other. 
In the center will be the pairing up of the numbers.
This gives us the below table.
\begin{center}
    \begin{tabular}{c| c c c c c c}
            &0  &1  &-1 &2  &-2 &\dots\\
        \hline
        $1$ &\{0,1\}    &\{1,1\}    &\{-1,1\}   &\{2,1\}    &\{-2,1\}   &\dots\\
        $2$ &\{0,2\}    &\{1,2\}    &\{-1,2\}   &\{2,2\}    &\{-2,2\}   &\dots\\
        $3$ &\{0,3\}    &\{1,3\}    &\{-1,3\}   &\{2,3\}    &\{-2,3\}   &\dots\\
        $4$ &\{0,4\}    &\{1,4\}    &\{-1,4\}   &\{2,4\}    &\{-2,4\}   &\dots\\
        \vdots&\vdots   &\vdots     &\vdots     &\vdots     &\vdots     &$\ddots$
    \end{tabular}
\end{center}

We can subsequently use the winding bijection on this to get our countably infinite set.
As a note, this can be done in turn with any pair of countably infinite sets paired together in some way, as long as they are mapped in terms of some function $f(a,b)$.

\pagebreak
\section{Exercise 2.10}
Let $S$ be the set of sequences ($a_n$) where, for  each $n, a_n \in \{0, 1\}$. Is $S$ countable or uncountable? 

\subsection*{Solution}
Note: Each sequence is infinite in this case.

Cantor's proof that $\mathbb{R}$ is uncountably infinite may be applicable here.
Suppose that we laid out an array of all the sequences as binary decimals less than 1.
\begin{align}
    &0.a_{11}a_{12}a_{13}a_{14}a_{15}\\
    &0.a_{21}a_{22}a_{23}a_{24}a_{25}\\
    &0.a_{31}a_{32}a_{33}a_{34}a_{35}\\
    &0.a_{41}a_{42}a_{43}a_{44}a_{45}\\
    &0.a_{51}a_{52}a_{53}a_{54}a_{55}
\end{align}

Suppose that we were to create a member of this set.
We could take every value of each member of each set such that the first designating number matches the second designating number (e.g. $a_{11}$, $a_{22}$, $a_{33}$, $a_{xx}$, \dots). 
For each value of that, we would create a formula for the respective value of member $b_i$.
\begin{equation}
    a_{ii} \to b_i = \begin{cases}
        1   &\text{if } a_{ii} = 0\\
        0   &\text{if } a_{ii} = 1
    \end{cases}
\end{equation}

This in turn creates a new value of a set in $S$, which is previously unwritten.
This in turn establishes that the set is uncountably infinite.

\end{document}