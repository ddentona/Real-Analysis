\documentclass[12pt]{report}
\usepackage{amsmath}
\usepackage{amssymb}
\usepackage{amsthm}


\begin{document}
\tableofcontents
\pagebreak
\chapter{}
\section{Exercise 1.1}
Explain the error in the following ``proof'' that 2 = 1.\\
Let $x = y$. Then
\begin{align*}
    x^2 &= xy\\
    x^2 + y^2 &= xy - y^2\\
    (x + y)(x - y) &= y(x - y)\\
    x + y &= y\\
    2y &= y\\
    2 &= 1
\end{align*}

\subsection*{Solution}
Between \((x + y)(x - y) = y(x - y)\) and \(x + y = y\), we divide by \(x - y\), which from the given, is equal to zero. Hence, the step involves dividing by zero, itself undefined and not allowed to be performed.

\pagebreak
\section{Exercise 1.2}
Which of the following statements are true? Give a short explanation for each of your answers?\\
(a) For every $n \in \mathbb{N}$ there is an $m \in \mathbb{N}$ such that $m > n$.\\
(b) For every $m \in \mathbb{N}$ there is an $n \in \mathbb{N}$ such that $m > n$.\\
(c) There is an $m \in \mathbb{N}$ such that for every $n \in \mathbb{N}$, $m \ge n$.\\
(d) There is an $n \in \mathbb{N}$ such that for every $m \in \mathbb{N}$, $m \ge n$.\\
(e) There is an $n \in \mathbb{R}$ such that for every $m \in \mathbb{R}$, $m \ge n$.\\
(f) For every pair $x < y$ of integers, there is an integer $z$ such that $x < z < y$.\\
(g) For every pair $x < y$ of real numbers, there is a real number $z$ such that $x < z < y$.

\subsection{Solution (a) True}
\begin{proof}
    We begin with our given, which is that $n \in \mathbb{N}$. 
    Natural numbers have a quality that they are ordered such that successive numbers are a distance of 1 apart.
    Natural numbers are also unlimited in the direction of positive infinity.
    This means that for every $n \in \mathbb{N}$, there is a greater natural number.
    In other words, for every $n \in \mathbb{N}$, there exists an $m \in \mathbb{N}$ such that $m > n$.
\end{proof}

\subsection{Solution (b) False}
To prove this false, we need but find a counterexample. 
For the counterexample, we look at the number 1. $1 \in \mathbb{N}$. However, there is no $n \in \mathbb{N}$ such that $1 > n$. As such, it is false.

\subsection{Solution (c) False}
There is not maximum number in the natural numbers. For this to be true, it would have to contradict part (a), which has been proven. As such it is false.

\subsection{Solution (d) True}
The contrapositive of this is there is no $n \in \mathbb{N}$ such that for every $m \in \mathbb{N}$, $m < n$. Since there are infinitely many $\mathbb{R}$ with no upward bound, there will be no number $n \in \mathbb{N}$ such that $m < n$ for every $m \in \mathbb{N}$. This makes it true.

\subsection{Solution (e) True}
The contrapositive of this is there is no $n \in \mathbb{R}$ such that for every $m \in \mathbb{R}$, $m < n$. Since there are infinitely many $\mathbb{R}$ with no upward bound, there will be no number $n \in \mathbb{R}$ such that $m < n$ for every $m \in \mathbb{R}$. This makes it true.

\subsection{Solution (f) False}
If $x$ and $y$ are adjacent, like 0 and 1, then the number $z$ such that $x < z < y$ would have to be of distance less than 1 from the two adjacent numbers. This contradicts the principle of integers, which says that all integers are of distance 1 from their adjacent numbers. The statement is false.

\subsection{Solution (g) True}
There are considered to be infinitely many real numbers between two other real numbers.
This means that for any $x, y \in \mathbb{R}$, there does exist $z \in \mathbb{R}$ such that $x < z < y$. 

\pagebreak
\section{Exercise 1.3}
If $A$ and $B$ are two boxes (possibly with things inside), describe the following in terms of boxes:
\begin{align*}
    &(a) A \setminus B 
    &(b) \mathcal{P}(A)
    &&(c) \left| A \right|
\end{align*} 

\subsection{Solution (a)}
$A \setminus B$ means the set of objects in A not in B. 
$A \setminus B = \{x: x \in A \text{and} x \notin B\}$. 
In terms of boxes, this would mean the box with everything that is in A but not in B.

\subsection{Solution (b)}
$\mathcal{P}(A)$ or the \textit{power set} of A means the set of all subsets of A.
\(\mathcal{P}(A) = \{X: X \subseteq B\}\).
In terms of boxes, this would be the box containing every box containing a combination of only different objects contained in $A$. 

\subsection{Solution (c)}
$\left| A \right|$ or the cardinatity of A means the count of all elements in $A$.
In terms of boxes, this would be the number of objects contained in the box $A$.

\pagebreak
\section{Exercise 1.4}
If $A_1, A_2, A_3, \dots, A_n$ are all boxes (possibly with things inside), describe the following in terms of boxes:
\begin{align*}
    &(a) \overunderset{n}{i = 1}{\bigcup} A_i = A_1 \cup A_2 \cup \dots \cup A_n
    &(b) \overunderset{n}{i = 1}{\bigcap} A_i = A_1 \cap A_2 \cap \dots \cap A_n
\end{align*}

\subsection{Solution (a)}
A box containing all the objects in each of the boxes $A_1, A_2, A_3, \dots, A_n$.

\subsection{Solution (b)}
A box containing each object that is in every one of the boxes $A_1, A_2,$ $A_3, \dots, A_n$.

\pagebreak
\section{Exercise 1.5}
Prove that each of the following holds for any sets $A$ and $B$.\\
(a) $A \cup B = A$ if and only if $B \subseteq A$.\\
(b) $A \cap B = A$ if and only if $A \subseteq B$.\\
(c) $A \setminus B = A$ if and only if $A \cap B = \emptyset$.\\
(d) $A \setminus B = \emptyset$ if and only if $A \subseteq B$.

\subsection{Solution (a)}
We have to prove $A \cup B = A$ if and only if $B \subseteq A$.
To prove this, we must prove that $A \cup B = A \implies B \subseteq A$ and $B \subseteq A \implies A \cup B = A$.
We begin with the first. 
\begin{proof}
    We know that $A \cup B = A$.
    This means that for any $b \in B$ or $b \in A$, $b \in A$. 
    Taking the first half of this, we get $b \in B \implies b \in A$.
    This can be rewritten as $B \subseteq A$. 
    QED
\end{proof}

Now we prove the second.
\begin{proof}
    We know that $B \subseteq A$.
    This can be written that $b \in A$ for every $b \in B$, or $\frac{b \in B}{b \in A}$.\\
    From identity, we also know that $b \in A$ for every $b \in A$, or $\frac{b \in A}{b \in A}$.\\
    Putting these together, we get $b \in A$ for every $b \in A$ or $b \in B$, rewriten as $\frac{(b \in A) \vee (b \in B)}{b \in A}$.\\
    Rewriting this in sets, we get $\frac{\{b : b \in A \text{ or } b \in B\}}{\{b: b \in A\}}$.\\
    Putting this in implies and sets notation, we finally end up with \(A \cup B = A\). \\
    QED
\end{proof}
Putting these together, we get the if and only if statement we were looking for. 

\subsection{Solution (b)}
We need to prove that $A \cap B = A$ if and only if $A \subseteq B$.
We can just prove that $A \cap B = A$ if $A \subseteq B$ and $A \subseteq B$ if $A \cap B = A$.
We begin with the first.
\begin{proof}
    (1) $A \subseteq B$ is given.
    From this, we know from definitional that for any $a \in A$, $a \in B$.\\
    \[a \in A \implies a \in B\].\\
    (2) We can also determine the existence of a set $C$ equal to $A \cap B$.
    From the definition, we know that for any $c \in C$, $c \in A$ or $c \in B$.
    \[c \in C \implies c \in A \text{or} c \in B\]
    \\
    (3) Plugging in the value from (1), we get 
\end{proof}



\end{document}